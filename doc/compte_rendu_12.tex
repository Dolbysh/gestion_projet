\documentclass{article}

\usepackage[utf8x]{inputenc}
\usepackage[french]{babel}
\usepackage{easylist}

\begin{document}

\begin{description} \parskip10pt
    \item[Le Mapping] \hfill \\
        \begin{itemize}
            \item Surtout quantifier la complexité et les probabilités
            \item Client pas d'accord sur la bitmap (au niveau de la place occupée : trop grande)
            \item Table de hashage à sauver périodiquement sur disque dur.
            \item Vérifier l'endroit où est "prise" la mémoire (physique, swap...)
        \end{itemize}

    \item[Le dialogue entre drivers] \hfill \\
        \begin{itemize}
            \item Résolution du problème sur l'obtention de l'identifiant d'un driver particulier.
            \item Recherches non concluantes : remplacement d'une fonction obsolète par
            une autre non immédiatement fonctionnelle. Impliquait un remaniement du code. 
        \end{itemize}

    \item[Plan d'action]\hfill  \\
        \begin{itemize}
            \item Contacter développeur du driver md (~2h) (baptiste)
            \item Exposés proposés par le client pour mieux appréhender les problèmes:
                \begin{itemize}
                    \item sur la structure et le fonctionnement d'un driver linux et du noyau(~8h)(baptiste et emmanuel)
                    \item sur le mapping et la solution apportée (~6h) (claire)
                \end{itemize}
        \end{itemize}

    \item[Sur le résultat du précédent plan d'action]\hfill \\
        \begin{itemize}
            \item Pas de réponses des mails ou réponses négatives.
            \item Quelques réponses sur les forums fr, mais pas satisfaisantes, dont l'utilisation
            de "fuse" (trop haut niveau), la simulation du "bcache", l'insertion d'une nouvelle
            "entrée" pour md (besoin d'étudier le code de certains modules md) et du "hack"
            du noyau.
        \end{itemize}
\end{description}
\end{document}
